\documentclass[sigconf]{acmart}

\usepackage{graphicx}
\usepackage{hyperref}
\usepackage{todonotes}

\usepackage{endfloat}
\renewcommand{\efloatseparator}{\mbox{}} % no new page between figures

\usepackage{booktabs} % For formal tables

\settopmatter{printacmref=false} % Removes citation information below abstract
\renewcommand\footnotetextcopyrightpermission[1]{} % removes footnote with conference information in first column
\pagestyle{plain} % removes running headers

\newcommand{\TODO}[1]{\todo[inline]{#1}}

\begin{document}
\title{What is the Role of Big Data in Health}


\author{Matthew Durbin, MD FAAP}
\orcid{1234-5678-9012}
\affiliation{%
  \institution{Indiana University School of Medicine Department of Pediatrics, Division of Neonatology
Riley Hospital for Children}
  \streetaddress{699 Riley Hospital Drive}
  \city{Indianapolis} 
  \state{Indiana} 
  \postcode{ 46202
}
}
\email{mddurbin@iu.edu}

% The default list of authors is too long for headers}
\renewcommand{\shortauthors}{M. Durbin}



\keywords{i523 \TODO{complete}}


\begin{abstract}
\TODO{MISSING}
\end{abstract}

\maketitle





\section{WHAT IS THE ROLE OF BIG DATA IN HEALTH}

The current state of healthcare system in the United States is often
described as a crisis.  The term comes with good reason, as spending
accounts for 17-18\% of GDP, dwarfing other nations, and is
exponentially rising at an unsustainable rate.  For all of our
spending, we have poorer health than most developed and many
developing nations.  The healthcare industry is behind in technology,
with recent adoption of an electronic medical record, and prior
reliance on paper charting.  Communication is most often by decades
old technology including phone or fax.  Internet communication between
healthcare providers, and with patients, is a recent novelty.  We have
to poorest health, including obesity due to poor diet, lack of
exercise, and substance abuse.  We pay more for pharmaceuticals than
any other country, and most pharmaceutical budget goes to marketing as
opposed to research and development.  Meanwhile the business world is
far ahead of healthcare.

Big Data has major potential to impact health.  Massive data sets
related to human health are compiled by insurance companies,
pharmaceutical companies, public health institutions and research
institutions.  Big data will soon have a huge impact on improving the
health, but there is a long road ahead. Much of the lag is due to
serious issues with privacy and security.  The healthcare industry
should be able to overcome these obstacles as online banking and
financial institutions have done.  There is amazing potential with big
data and healthcare, but a long way to travel. \cite{fox6} Healthcare is
making strides and big data collection is visible everywhere.  The
electronic medical record EMR is close to universal and is improving
constantly.  Medical resources are accessible around the world through
smartphones, making medical libraries obsolete.  Next generation
sequencing technologies are able to measure the genetic contributions
to disease that previously a mystery. Wearable technology and fitness
tracking apps, nutrition apps are improving personal.

\section{COST OF HEALTHCARE}


\subsection{The Current State}

One of the most troubling issues facing the United States, and the
world, is the increasing cost of healthcare.  The problems are
different around the globe.  Much of the developing world lacks access
to adequate healthcare, which is a serious problem. This paper focuses
on a different problem, in the crisis facing the United
States. Current healthcare spending is greater than 3 trillion dollars
\cite{centers2014national}.  This makes up 17 percent of GDP.  This number grows every year
and is unsustainable.  This number affects citizens deeply, and
currently healthcare costs are responsible for 50\% of bankruptcy
claims in the United States \cite{fox6}. All of this extra spending does not
equal better health.  In most measures of health, from infant
mortality to life expectancy, the United States find itself far from
the top.  There are major issues at play ranging from a massive
bureaucracy, to the poor health and obesity of participants.

\subsection{The Future}

It is projected that the average family will spend over 25\% of income
on to healthcare \cite{fox6}.  The problem is not projected to improve.  As
the “baby-boomers” age, the population over 60 with high cost chronic
healthcare problems, increases exponentially.  In Medical School, we
were taught about this “silver tsunami” approaching the US healthcare
system (prompting me to go into Pediatrics.)  Many individuals,
including myself, look to Big Data to uncover these problems and help
fix them. Before it is too late.  There are technology solutions
including the electronic health record, medical reference technology,
genomic medicine, telemedicine, wearable health technology, and
personalized medicine.

\section{ELECTRONIC HEALTH RECORD}

\subsection{Adoption of and EMR}

Throughout history, medical records were taken on paper, but after 2000 the slow transition to electronic records began \cite{kokkonen2013use}. The handwritten records were kept in large file cabinets, and when records needed to be shared between physicians or institutions (across the country or across the street), the paper records were faxed over a telephone line.  This technology is decades old.  As technology raced forward with supercomputers and the worldwide web, medicine continued to use these antiquated forms of communication.  Finally, government mandating forced healthcare systems into the modern era and electronic records went online.  Currently over 84\% of health records are online \cite{fox6}.  

\subsection{	The Current State}

A majority of healthcare systems around the world are under a government regulated socialized medical system which comes with a universal health record. The healthcare system in the United states is privatized, therefore the transition to EHR came with individual health entities purchasing a multitude of different EHRs.  The problem comes in that a patient presenting to two different healthcare facilities, even if across the street or within the same building, will have two different medical charts that do no communicate with one another.  
The other problem comes with accessing this information.  The two largest companies Epic and Cerner have a commercial interest, with a primary goal to increase revenue to the shareholder.  It is exceedingly difficult for the nonprofit entities including academic centers and hospitals to access the patient information within the EHR. There is tremendous potential within the EHR.  Beyond data collection, storage, data retrieval, and analysis, we should move towards real time guidance and guidelines for medical decision making to improve health.   
 
\section{KNOWLEDGE}

Only 10-20 years ago, Hospital libraries and medical school libraries were once filled with books and journal articles.  If a healthcare practitioner wanted information relevant to clinical care, they went to libraries to pour through the resources with exhaustive efforts.   Today, those libraries are mostly void of books.  Almost every individual in western medicine has access to a computer, and usually to a handheld device, capable of accessing far more information than could ever be stored in a library.  There are massive information sources, such as PubMed, a gigantic repository of journal articles and books that is constantly being updated with new information.  And Up To Date, a point of care medical reference commonly used on a handheld deceive, with evidence based clinical guidelines contributed by over 5,000 physicians \cite{wiki-uptodate}. The massive amount of data now accessible to most healthcare providers and scientists is changing healthcare rapidly.  Still, there is much room for improvement as care is commonly delivered based on anecdotal evidence, and cost and quality should continue to improve.    

\section{NEXT GENERATION SEQUENCING}

\subsection{	The Human Genome}

The first human genome was sequenced in 2003\cite{collins2003human}.  This colossal global
effort took over 10 years and thousands of scientists working at great
expense.  In the end, a private and public group collectively
sequenced the first genome.  Initially, the technology was extremely
expensive and took great deal of time.  Through technological
advancements including sequencing cores and big data, the cost of the
genome has plummeted.  The 1000-dollar genome project is an attempt to
make sequencing more affordable \cite{fox6}.  We are a long way away form
being able to utilize the genome to deliver care.  Bioinformatics
expertise has lagged behind technology.  Groups still do not agree on
a standard way to process the information.  Still this technology
improves papidly, and recently a group published 24-hour genome
sequencing for intended us in clinical decision making. Soon it may be
a reality for physicians to utilize genomic information, whether about
drug susceptibility, or prognosis, to guide medical care.

\subsection{	Beyond DNA}

Initial estimates placed the number of genes at >100,000 \cite{vanderbilt}.  Looking
at the massive amount of diversity and the billions of unique human
beings on this earth, this was a appropriate estimate.  The current
number is estimated somewhere around 20,000.  The question is what
accounts for the rest of phenotypic diversity and disease.  The human
genome project utilized whole exome sequencing.  Whole exome
sequencing involves sequencing the entire coding region, or exome, of
the genome.  This consists of around 20,000 genes and over 30 million
nucleotides.  The exome, though massive, consists of only 1\% of the
total genomic DNA.  Many genetic diseases involve alteration of this
coding exome but we are discovering that many diseases are due to
problems outside of this coding region.  Sequencing only 1\% of the
genomic material is a fraction of the time, cost, and burden of
analysis, compared with whole genome sequencing, but we must move
towards whole genome sequencing to capture all disease states.  We
have also come to realize that splicing and other post transactional
regulation introduces much diversity.  We have the technology to
sequence the entire RNA transcriptome and the proteome as well.  This
produces a data set which dwarfs the genome and genomic DNA sequence
information.  These technologies are currently only utilized in the
research setting.  Despite our advanced technology, we have very
little idea of how to interpret the data in a clinical setting.  Again
the bioinformatics expertise lags behind.  There is amazing potential
to advance knowledge and study human disease and a tremendous amount
of big data analytics along the way.

\section{WEARABLE TECHNOLOGY, NURITION AND WELLNESS APPS}

Massive data sets exist, collected by insurance companies, in
electronic health records, by pharmaceutical companies and by research
institutions.  There is another very exciting source of big data on
the horizon, in personal wearable technologies, and also fitness,
wellness and nutrition apps \cite{fox6}.  Individuals wearing FitBits, with
fitness apps on their mobile devices, wearing smartwatches, etc. can
track health and wellness measures in ways that once required
inpatient hospital monitoring and sophisticated research lab settings.
They track sleep and activity throughout the day and night.  In
addition, there are countless apps which track nutrition and health.
People log meals and nutrition to keep accountable.  Often these apps
work with time tested and well researched diets including weight
watchers, etc.  This technology has already changed the way many
individuals look at health and wellness.  This exciting new dataset
has great potential to advance human health and improve disease that
may be the root cause of our healthcare epidemic.

\section{TELEMEDICINE}

Telemedicine involves a virtual visit between a physician and patient
\cite{hernandez2016pediatric}.  There are obvious benefits, especially when a patient population
is spread across a wide geographic space either due to a high level of
physician specialization, or a rural patient population. Highly
specialized, but critical subspecialists are often in great shortage.
This places a great burden on the available providers, with often
unsustainable schedules.  Video technology allows doctors, nurses and
practitioners to visualize patients, perform a limited physical, and
to communicate with individuals at a distance.  There is great
potential to improve cost and reduce burden. There are limitations.
Many physician specialists are values for their technical, hands on
skills.  Telemedicine is not much of a help, the technical procedures,
such as inserting airways into the trachea of small babies, and insert
central arterial lines into major vessels to deliver lifesaving
medications, require hands on skills.  The same goes for surgeons and
other highly skilled technical professions.  Interventional techniques
and robotics are increasingly being used to perform procedures, but
while these operations are performed, a surgeon needs to very close,
in case unforeseen accidents problems necessitate a conventional
correction. Procedural specialties are the greatest expense to our
healthcare system and their procedural skills are a long way from
being performed through telemedicine or robotics.
 
\section{SOCIAL MEDIA}

One interesting trend is the multitude of health information shared
over social media networks.  Blogs, columns, and posts providing
information about nutrition and wellness, news stories, and
information sharing.  The story reporting google’s flu prediction
trends ahead of the CDC, based on search history, spread virally over
facebook \cite{ginsberg2009detecting}.  The field will continue to expand.  wonderfully

\section{PERSONALIZED MEDICINE}

Wikipedia summarized personalized medicine as: “a medical
procedure that separates patients into different groups—with medical
decisions, practices, interventions and/or products being tailored to
the individual patient based on their predicted response or risk of
disease.”  \cite{wiki-personalized} In a way the culmination of big data and health is with
personalized medicine.  In a hopefully not so distant future the
electronic health record, pharmaceutical data and genomic data will
provide a more tailored, affordable, and high-quality approach to
healthcare.  Hopefully healthcare will catch up with financial and
ecommerce and in their ability to harness big data for good.


\section{	CONCLUSION}

This paper highlights just a handful of technology driven big data
solutions to our healthcare crisis.  As Congress debates legislature
to face this crisis, big data more harmoniously moves towards
solutions.  Better health without economic ruin is a reality and big
data will play a major role.  Much work is left to be done

\begin{acks}

Thank you to Dr. Geoffrey Fox, Gregor von Laszewski, and all of the
course instructors for an excellent introduction to Big Data and Data
Science.

\end{acks}

\section{	REFERENCES}

\TODO{
Unit 6 Lectures by Geoffrey Fox
Centers for Medicare \& Medicaid Services. "National health expenditures 2012 highlights." Online verfügbar unter http://www. cms. gov/Research-Statistics-Data-and-Systems/Statistics-Trends-and-Reports/National-HealthExpendData/Downloads/highlights. pdf (2014).
Kokkonen, Erik WJ, et al. "Use of electronic medical records differs by specialty and office settings." Journal of the American Medical Informatics Association 20.e1 (2013): e33-e38.
"UpToDate." Wikipedia: The Free Encyclopedia. Wikimedia Foundation, Inc. 22 July 2004. Web. 2 Sept. 2016.
Collins, Francis S., Michael Morgan, and Aristides Patrinos. "The Human Genome Project: lessons from large-scale biology." Science 300.5617 (2003): 286-290.
Hernandez, Maria, et al. "Pediatric critical care telemedicine program: A single institution review." Telemedicine and e-Health 22.1 (2016): 51-55.
Ginsberg, Jeremy, et al. "Detecting influenza epidemics using search engine query data." Nature 457.7232 (2009): 1012-1014.
"Personalized Medicine." Wikipedia: The Free Encyclopedia. Wikimedia Foundation, Inc. 22 July 2004. Web. 2 Sept. 2016.
Vanderbilt University: Introduction to Bioinformatics Course Lectures
}

size \cite{editor00}.





\bibliographystyle{ACM-Reference-Format}
\bibliography{report} 

\end{document}
