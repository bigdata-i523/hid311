\documentclass[sigconf]{acmart}

\usepackage{graphicx}
\usepackage{hyperref}
\usepackage{todonotes}

\usepackage{endfloat}
\renewcommand{\efloatseparator}{\mbox{}} % no new page between figures

\usepackage{booktabs} % For formal tables

\settopmatter{printacmref=false} % Removes citation information below abstract
\renewcommand\footnotetextcopyrightpermission[1]{} % removes footnote with conference information in first column
\pagestyle{plain} % removes running headers

\newcommand{\TODO}[1]{\todo[inline]{#1}}

\begin{document}
\title{What is the Role of Big Data for Wildlife Conservation and Resource Management}


\author{Matthew Durbin, MD FAAP}
\orcid{1234-5678-9012}
\affiliation{%
  \institution{Indiana University School of Medicine Department of Pediatrics, Division of Neonatology
Riley Hospital for Children}
  \streetaddress{699 Riley Hospital Drive}
  \city{Indianapolis} 
  \state{Indiana} 
  \postcode{ 46202
}
}
\email{mddurbin@iu.edu}

% The default list of authors is too long for headers}
\renewcommand{\shortauthors}{M. Durbin}



\keywords{i523{Big Data,farming, agriculture, species reintroduction, environment, conservation}}


\begin{abstract}
\TODO{MISSING}
\end{abstract}

\maketitle





\section{THE ROLE OF BIG DATA IN WILDLIFE CONSERVATION}
\subsection{Introduction}
The human population continues to grow exponentially, from 1 billion in 1800 to 7 billion in 2012.  A primary concern is lifting the standard for humans in the fastest growing regions, often affected by the worse living conditions.  Along with that comes, sustaining our global environment.  The flora and fauna which make our world habitable are increasingly destroyed as human populations expand, and this affects those disenfranchised humans the most.  Vital processes to human life are sustained by the environment, including fertilization, waste breakdown, and oxygen and carbon dioxide exchange.  Understandably, Wildlife conservation has been a crucial pursuit for hundreds of years. Scientists collect massive amounts of data based on experiments and observation and analyze that data to make discovery.  Wildlife Biologists are scientists who focus on wildlife management and conservation. Before advances in technology, wildlife biologists would sit for long hours counting or tracking the movement animals, and making observations about habitat.  Jane Goodall famously spent years living amongst and observing Chimpanzees to make her landmark contributions [2]. Big data has had a major impact on all ways of life, and especially in the advancement of science.  Big Data has been called on to right our spiraling healthcare system, balance our financial system after the mortgage meltdown, and now to save our planet. Wildlife Biologists are increasingly relying on technology and Big Data  analytics to aide conservation.   Tech Republic published an article in November 2014 about big data projects that could help save the planet. The article outlines 10 exciting projects using big data for wildlife conservation [1]. I will focus on wildlife tracking and data visualization.

\subsection{1.2	Wildlife tracking}
The life of a wildlife Biologist like Jane Goodall once involved long hours in extreme environmental conditions counting animals.  Their great knowledgebase could have been used to analyze data and make advancements and contributions, but were often unable to reach that point.  Modern animal monitoring and tracking systems may have contributed immensely.  Jane Goodall made a landmark contribution to science with her observations of chimpanzees [2].  Imagine the contribution she would make with todays animal tracking technologies!

\subsection{    An app for Bird Tracking}
An interesting project discussed in the TechRepblic Article was the Nature Conservatory use of an app called eBird [3] utilizing NASA satellite data to track migrating birds.  Scientists use the data about the birds path to track progress, and also to try and help alter course or create habitat when it has been destroyed by human activity.  One interesting concept, was when marshland had been destroyed, they created habitat by paying farmers to flood their fields (a practice that also benefited farmers through increased soil fertility.) 
\subsection{    Intel Chip and 3G networks for Rhino} Tracking
Another interesting story was how Intel South Africa built a custom chip to help track critically endangered rhinoceroses.  The chip utilizes Vodaphone’s 3G network connection and a cloud company called Dimension Data. The ankle bracelet alerts conservation workers of the rhinos whereabouts. A second chip located in the horn alerts anti poaching teams when the horn is removed and separated from the ankle bracelet.  
\subsection{    Mountain Lion SMART Collar}
The California Mountain Lion is a very elusive animal and very difficult to spot in the wild and therefore study.  The animals habitat is being destroyed at rapid pace as development and population in California booms.  Biologists have have developed a collar to study the habits of this seldom seen animal, including how it catches prey [4].  Their efforts and findings made it into Science, one of, if not the top scientific journal in the world.  Hopefully their efforts can save this magnificent animal. 
\subsection{    Citizen Science Project: Bumble Bee Watch}
A great deal of data points to the declining bumble bee populations around the world.  This is extremely concerning for human beings, because bumblebees pollinate almost all the worlds plants, including food sources, and the practice that is completely un-replicable by humans.  One project that is helping immensely is the Bumble Bee Watch.  This is a crowdsourcing efforts undertaken by individuals around the United States to log bumble bees, when sited, and the data is reviewed and utilized by scientists.  This crowdsourcing project is being termed a: “citizen science project.”       

\subsection{Species Reintroduction}
In Yellowstone National park, Wolves had been hunted to extinction by humans in the 1920s.  In the summer of 1995, the a pack of 66 grey wolves was reintroduced to the park to balance the fragile ecosystem, previously missing a top predator.  I The project has achieved success beyond plans.  Big data has been used extensively during the wolf reintroduction and monitoring.  Examples include initial analysis of how many wolves to introduce and where to release the wolf pack, tracking the wolf pack and elk herd through tracking collars, satellites, and and visitor reports, monitoring plant ecosystems, and tracking the economic impact for ranchers whose cattle fall prey to wolf attacks, etc.    All of these projects utilized big data to monitor this extremely complex ecosystem of human, plant and animal interaction. There are very interesting papers most of which use big data to study the ecosystem, including how the wolf rates directly related to elk herd biomass[6-9].  
The project has been a great success.  The wolf pack currently numbers more than 300.  The once massive, single, unchallenged, elk herd weighed an average of over 700 lbs per animal.  The herd decimated new growth plants in the park wherever it went.  The heard, has split into groups, herd movement has drastically increased, and the elk are a healthier size and weight. Because the heard moves more, they do not decimate single plant populations as they once did, and now graze in one area for a short time, before fleeing the wolf pack.  Multiple plant species have rebounded as the wolf pack changed elk herd behavior.  Rare wildflowers and plants are blooming in mass, once again.  This has helped many small mammals thrive, which has allowed birds of prey to rebound.  
Aspen trees have rebounded drastically, as have many of the birds and insects that live in Aspen trees.  Another plant species to rebound includes the prized willow stands along Yellow Stone rivers.  The rebounding willow stands have led to a resurgence of another species previously near extinction, the Beaver.  The population of Yellowstone beavers has rebounded as a side effect of wolf reintroduction, elk herd behavior change, and willow stand resurgence. The introduction of Beaver and their dams has changed the course of the Yellowstone River.  The reestablishment of the true river course has refilled many small ponds and multiple fish, amphibian and reptile species have also recovered. 
The entire complex ecosystem is rebalanced with the introduction of the wolf, and these papers use big data to track the progress.  The current state can be checked in real time at the fish and wildlife service website [10].    

\subsection{1.4	Data Visualizations}
Social media has been used increasingly to share information on a massive scale. The interconnection we now experience through social media brings the opportunity to face many new and broad topics.  From political campaigns to wildlife conservation, the goal is often education and emotional arousal. Wildlife education content is perfectly designed for viewing on social media.  It brings an often geographically distant subject to the forefront.  The content often comes from media outlets like CNN, political groups like PETA, and other sources.  Optimal data visualization becomes important, in bringing an issue to light, to someone with limited background, in a short period of time, with competition from many other sources.  
\subsection{    Periscopic: The State of The Polar Bear}
Periscopic is a data visualization program is used to educate citizens about polar bears.  The project is maintained by scientists studying polar bears in the artic.  It contains information about the polar bear populations, their habitat, mating patterns, food source, pollution, dangers they face, the shrinking habitat, and human influence.  This is a unique opportunity for citizens thousands of miles away to have updated information about polar bears and issues they face.  The data visualization is excellent, providing a valuable education about this important topic to busy individuals.        



\section{2.	WHAT IS THE ROLE OF BIG DATA IN FARMING}


\subsection{Introduction}
The exploding population means a great increase in food production. Increasing food production requires increasingly efficient and sustainable methods, and Big Data has a major role. Farming and agriculture have become a large industrialized business in the United States.  Like most large profitable industries, big data has come to drastically influence the field.  The focus is on increasing productivity and decreasing costs.  Most universities offer courses and degrees in the topic.  There is an academic journal titled “Computers and Electronics in Agriculture.”  A short peruse of the articles offers a glimpse of the use of big data in animal farming, from egg production, to the meat industry, to dairy farming. A recent review article in the  journal Big Data and Society, discussed some of the recent advances in big data in farming and agriculture [11].  There are countless examples of Big Data and farmoing.  John Deere farm equipment utilizes GPS signaling for movement around a fields and all of the movement is collected and stored.  In addition the computers take soil and water samples to track the status of soil and water across the globe.  All of the information is kept in a proprietary database.  The information is extremely valuable to learn about the status of agriculture in the united states.  Monsanto has a weed ID app that helps individuals identify a weed and recommends a tailored pesticide for is killing.  All of these issues have major implications.  Beyond agriculture, animal farming is also influenced extensively by big data.  One interesting example includes the use of big data in precision dairy farming.  The practice of precision dairy farming uses big data to analyze the the process of production from cow in the pasture to milk and cheese on the store shelf[12]. 
\subsection{Data Modeling for Precision Dairy Farming}
Dairy farming is a massive and lucrative business in the United States.  Precision dairy farming has become an important field of study as the intersection between science, technology and dairy farming. Precision dairy farming utilizes Data collection and data modeling including multi-dimensional entity-relationship models [12].   The Precision dairy farming organization is a huge international organizations that hold three major conferences each year, in the United States, and internationally.  At the conferences the most recent research and technology is presented.   The conferences are sponsored by large corporations and by the federal government [13]. 

\section{CONCLUSION}
As the population continues to grow, we will continue to utilize and increasing amount of resources.  Optimal utilization of these resources is the only way to ensure survival and proper living standard for the human population.  Big data plays a major role in efficient utilization of resources including wildlife management, land management and agriculture.  Big Data is plays an increasing role in sustaining and improving our world.  

\begin{acks}

Thank you to Dr. Geoffrey Fox, Gregor von Laszewski, and all of the
course instructors for an excellent introduction to Big Data and Data
Science.

\end{acks}


\bibliographystyle{ACM-Reference-Format}
\bibliography{report} 

\end{document}
